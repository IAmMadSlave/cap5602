%----------------------------------------------------------------------------------------
%	PACKAGES AND OTHER DOCUMENT CONFIGURATIONS
%----------------------------------------------------------------------------------------

\documentclass{article}

\usepackage{fancyhdr} % Required for custom headers
\usepackage{lastpage} % Required to determine the last page for the footer
\usepackage{extramarks} % Required for headers and footers
\usepackage{graphicx} % Required to insert images
\usepackage{listings}
\usepackage{color}
\usepackage{xcolor}
\usepackage{caption}
\usepackage{enumitem}
\usepackage{amsmath}
\usepackage{tikz}
\usetikzlibrary{calc,shapes.multipart,chains,arrows}

\DeclareCaptionFont{white}{\color{white}}
\DeclareCaptionFormat{listing}{%
\parbox{\textwidth}{\colorbox{gray}{\parbox{\textwidth}{#1#2#3}}\vskip-4pt}}
\captionsetup[lstlisting]{format=listing,labelfont=white,textfont=white}

% Margins
\topmargin=-0.45in
\evensidemargin=0in
\oddsidemargin=0in
\textwidth=6.5in
\textheight=9.0in
\headsep=0.25in 

\linespread{1.1} % Line spacing

% Set up the header and footer
\pagestyle{fancy}
\lhead{\hmwkAuthorName} % Top left header
\chead{\hmwkClass\ (\hmwkClassInstructor\ \hmwkClassTime): \hmwkTitle} % Top center header
\rhead{\hmwkDueDate} % Top right header
\lfoot{\lastxmark} % Bottom left footer
\cfoot{} % Bottom center footer
\rfoot{Page\ \thepage\ of\ \pageref{LastPage}} % Bottom right footer
\renewcommand\headrulewidth{0.4pt} % Size of the header rule
\renewcommand\footrulewidth{0.4pt} % Size of the footer rule

\setlength\parindent{0pt} % Removes all indentation from paragraphs

%----------------------------------------------------------------------------------------
%	DOCUMENT STRUCTURE COMMANDS
%	Skip this unless you know what you're doing
%----------------------------------------------------------------------------------------

\setcounter{secnumdepth}{0} % Removes default section numbers
\newcounter{homeworkProblemCounter} % Creates a counter to keep track of the number of problems

\newcommand{\homeworkProblemName}{}
\newenvironment{homeworkProblem}[1][Problem \arabic{homeworkProblemCounter}]{ % Makes a new environment called homeworkProblem which takes 1 argument (custom name) but the default is "Problem #"
\stepcounter{homeworkProblemCounter} % Increase counter for number of problems
\renewcommand{\homeworkProblemName}{#1} % Assign \homeworkProblemName the name of the problem
\section{\homeworkProblemName} % Make a section in the document with the custom problem count
}

%----------------------------------------------------------------------------------------
%   COLORS AND LANGUAGAGE
%----------------------------------------------------------------------------------------

\lstset{
    frame=lrb,xleftmargin=\fboxsep,xrightmargin=-\fboxsep,language=Java,basicstyle=\ttfamily,
    breaklines=true,columns=fullflexible,keepspaces=true,escapeinside={\%*}{*)}
       }

%----------------------------------------------------------------------------------------
%	NAME AND CLASS SECTION
%----------------------------------------------------------------------------------------

\newcommand{\hmwkTitle}{Homework\ \#02} % Assignment title
\newcommand{\hmwkDueDate}{Tuesday,\ September\ 8,\ 2015} % Due date
\newcommand{\hmwkClass}{CAP\ 5602} % Course/class
\newcommand{\hmwkClassTime}{5:00pm} % Class/lecture time
\newcommand{\hmwkClassInstructor}{Finlayson} % Teacher/lecturer
\newcommand{\hmwkAuthorName}{Musa V. Ahmed} % Your name

%----------------------------------------------------------------------------------------

\begin{document}
\belowcaptionskip=-10pt

%----------------------------------------------------------------------------------------
%	PROBLEM 1
%----------------------------------------------------------------------------------------

\begin{homeworkProblem}
    3.14 Which of the following are true and which are false? Explain your answers.
    \begin{enumerate}[label=\alph*]
        \item Depth-first search always expands at least as many nodes as $A^*$ search with
            an admissible heuristic.
        \item $h(n)=0$ is an admissible heuristic for the 8-puzzle.
        \item $A^*$ is of no use in robotics because precepts, states, and actions are 
            continuous.
        \item Breadth-first search is complete even if zero step costs are allowed.
        \item Assume that a rook can move on a chessboard any number of squares in a 
            straight line, vertically or horizontally, but cannot jump over other pieces. 
            Manhattan distance is an admissible heuristic for the problem of moving the
            rook from square A to square B in the smallest number of moves.
    \end{enumerate}

\end{homeworkProblem}
\clearpage

%----------------------------------------------------------------------------------------

%----------------------------------------------------------------------------------------
%	PROBLEM 2
%----------------------------------------------------------------------------------------

\begin{homeworkProblem}
    3.15 Consider a state space where the start state is number $1$ and each state $k$ has two 
    successors: numbers $2k$ and $2k+1$.
    \begin{enumerate}[label=\alph*]
        \item Draw the portion of the state space for states $1$ to $15$.
        \item Suppose the goal state is $11$. List the order in which nodes will be visited 
            for breadth-first search, depth-limited search with limit $3$, and iterative 
            deepening search.
        \item How well would bidirectional search work on this problem? What is the 
            branching factor in each direction of the bidirectional search?
        \item Does the answer to (c) suggest a reformulation of the problem what would 
            allow you to solve the problem of getting from state $I$ to a given goal state 
            with almost no search?
        \item Call the action going from $k$ to $2k$ left, and the action going to $2k+1$ right. 
            Can you find an algorithm that outputs the solution to this problem without 
            any search at all?
    \end{enumerate}

\end{homeworkProblem}
\clearpage

%----------------------------------------------------------------------------------------

%----------------------------------------------------------------------------------------
%	PROBLEM 3
%----------------------------------------------------------------------------------------

\begin{homeworkProblem}
    3.18 Describe a state space in which iterative deepening search performs much worse 
    than depth-first search (for example, $\mathcal{O}(n)$ vs. $\mathcal{O}(n)$).

\end{homeworkProblem}
\clearpage

%----------------------------------------------------------------------------------------

%----------------------------------------------------------------------------------------
%	PROBLEM 4
%----------------------------------------------------------------------------------------

\begin{homeworkProblem}
    3.21 Prove each of the following statements, or give a counterexample:
    \begin{enumerate}[label=\alph*]
        \item Breadth-first search is a special case of uniform-cost search.
        \item Depth-first search is a special case of best-first tree search.
        \item Uniform-cost search is a special case of $A^*$ search.
    \end{enumerate}

\end{homeworkProblem}
\clearpage

%----------------------------------------------------------------------------------------

%----------------------------------------------------------------------------------------
%	PROBLEM 5
%----------------------------------------------------------------------------------------

\begin{homeworkProblem}
    3.23 Trace the operation of $A^*$ search applied to the problem of getting to Bucharest 
    from Lugoj using the straight-line heuristic. That is, show the sequence of nodes that 
    the algortihm will consider and the $f, g, $ and $Li$ score for each node.

\end{homeworkProblem}
\clearpage

%----------------------------------------------------------------------------------------


\end{document}
