%----------------------------------------------------------------------------------------
%	PACKAGES AND OTHER DOCUMENT CONFIGURATIONS
%----------------------------------------------------------------------------------------

\documentclass{article}

\usepackage{fancyhdr} % Required for custom headers
\usepackage{lastpage} % Required to determine the last page for the footer
\usepackage{extramarks} % Required for headers and footers
\usepackage{graphicx} % Required to insert images
\usepackage{listings}
\usepackage{color}
\usepackage{xcolor}
\usepackage{caption}
\usepackage{enumitem}
\usepackage{amsmath}
\usepackage{tikz}
\usetikzlibrary{calc,shapes.multipart,chains,arrows}

\DeclareCaptionFont{white}{\color{white}}
\DeclareCaptionFormat{listing}{%
\parbox{\textwidth}{\colorbox{gray}{\parbox{\textwidth}{#1#2#3}}\vskip-4pt}}
\captionsetup[lstlisting]{format=listing,labelfont=white,textfont=white}

% Margins
\topmargin=-0.45in
\evensidemargin=0in
\oddsidemargin=0in
\textwidth=6.5in
\textheight=9.0in
\headsep=0.25in 

\linespread{1.1} % Line spacing

% Set up the header and footer
\pagestyle{fancy}
\lhead{\hmwkAuthorName} % Top left header
\chead{\hmwkClass\ (\hmwkClassInstructor\ \hmwkClassTime): \hmwkTitle} % Top center header
\rhead{\hmwkDueDate} % Top right header
\lfoot{\lastxmark} % Bottom left footer
\cfoot{} % Bottom center footer
\rfoot{Page\ \thepage\ of\ \pageref{LastPage}} % Bottom right footer
\renewcommand\headrulewidth{0.4pt} % Size of the header rule
\renewcommand\footrulewidth{0.4pt} % Size of the footer rule

\setlength\parindent{0pt} % Removes all indentation from paragraphs

%----------------------------------------------------------------------------------------
%	DOCUMENT STRUCTURE COMMANDS
%	Skip this unless you know what you're doing
%----------------------------------------------------------------------------------------

\setcounter{secnumdepth}{0} % Removes default section numbers
\newcounter{homeworkProblemCounter} % Creates a counter to keep track of the number of problems

\newcommand{\homeworkProblemName}{}
\newenvironment{homeworkProblem}[1][Problem \arabic{homeworkProblemCounter}]{ % Makes a new environment called homeworkProblem which takes 1 argument (custom name) but the default is "Problem #"
\stepcounter{homeworkProblemCounter} % Increase counter for number of problems
\renewcommand{\homeworkProblemName}{#1} % Assign \homeworkProblemName the name of the problem
\section{\homeworkProblemName} % Make a section in the document with the custom problem count
}

%----------------------------------------------------------------------------------------
%   COLORS AND LANGUAGAGE
%----------------------------------------------------------------------------------------

\lstset{
    frame=lrb,xleftmargin=\fboxsep,xrightmargin=-\fboxsep,language=Java,basicstyle=\ttfamily,
    breaklines=true,columns=fullflexible,keepspaces=true,escapeinside={\%*}{*)}
       }

%----------------------------------------------------------------------------------------
%	NAME AND CLASS SECTION
%----------------------------------------------------------------------------------------

\newcommand{\hmwkTitle}{Homework\ \#01} % Assignment title
\newcommand{\hmwkDueDate}{Tuesday,\ September\ 1,\ 2015} % Due date
\newcommand{\hmwkClass}{CAP\ 5602} % Course/class
\newcommand{\hmwkClassTime}{5:00pm} % Class/lecture time
\newcommand{\hmwkClassInstructor}{Finlayson} % Teacher/lecturer
\newcommand{\hmwkAuthorName}{Musa V. Ahmed} % Your name

%----------------------------------------------------------------------------------------

\begin{document}
\belowcaptionskip=-10pt

%----------------------------------------------------------------------------------------
%	PROBLEM 1
%----------------------------------------------------------------------------------------

\begin{homeworkProblem}
    1.2 Read Turing's original paper on AI (Turing, 1950). In the paper, he discusses several 
    objections to his proposed enterprise and his test for intelligence. Which objections still carry 
    weight? Are his refutations valid? Can you think of new objections arising from developments 
    since he wrote the paper? In the paper, he predicts that, by the year 2000 a computer 
    will have a 30\% chance of passing a five-minute Turing Test with an unskilled interrogator. 
    What chance do think a computer would have today? In another 50 years?
    \\
    I personally believe that all of the objections brought forth in Turing's paper are still valid 
    to some degree today. I will now briefly describe why for each objection. The Theological Objection 
    still stands because religion is still a major part of society and still influences discourse 
    concerning any kind of 'life'. The 'Heads in the Sand' Objection may not be as prevalent as it was 
    in Turing's time, however, I am confident many people still hold the opinion that men will always 
    be superior. I think the Mathematical Objection will continue to hold while we are still using 
    traditional computing systems. Perhaps the Mathematical Objection will be invalid once we transfer 
    over to newer model of computing (e.g., one based on the human brain). The Argument from Consciousness 
    is among the weakest of the objections today. Computers can already compose beautiful pieces of 
    music (e.g., Emily Howell), but do they computer understand what they have done? That question 
    still remains open. Arguments from Various Disabilities are very interesting to thinking about 
    in todays age. Computers can perform nearly every single task a human can, however, the objection 
    still holds because computers are not capable of emotion. Lady Lovelace's Objection still holds 
    today as well, consider all modern day computing, despite popular belief every single behavior 
    that a computer has is a result of some kind of programming. One can argue that no matter how 
    complex the behavior observed in the computer it is merely following orders. A counterpoint to 
    this object would be that a child does as asked until he or she grows old enough to make their 
    own decisions. The Argument from Continuity in the Nervous System is hottest objection of our time. 
    A popular belief on the discovering of AI is that it will not be possible until we understand the 
    human brain better. Only then can we construct machines which rival the brain. The Argument from 
    Informality of Behavior is still valid but on shaky ground. Computers have become extremely capable 
    inferring and making decisions with everyday choices, however, there is still a lot of ground to be 
    covered. Computer still can not have 'moral code', they make have some base rules for morality but 
    I would argue this is not the same. The Argument from Extra-Sensory Perception is strange in that 
    at first it seems quite ridiculous, I do not believe we can completely dismiss the concerns it 
    until we either completely prove or disprove E.S.P.
    \\
    Turing's refutations are valid for the most part, however, one could argue that his refutations 
    for The Theological Objection and The 'Heads in the Sand' Objection are completely subjective. In 
    fact after closer analysis while most of refutations are well thought out a decent portion of them 
    are based on conjecture. 
    \\
    Computers with a combination of pre-programmed responses and actual responses exist today and 
    I believe can pass a five-minute Turing Test with an unskilled interrogator. In 50 years, it may 
    possible that all computers can completely pass the Turing Test, I do not believe to be likely.

\end{homeworkProblem}
\clearpage

%----------------------------------------------------------------------------------------

%----------------------------------------------------------------------------------------
%	PROBLEM 2
%----------------------------------------------------------------------------------------

\begin{homeworkProblem}
    1.14 Examine the AI literature to discover whether the following tasks can currently be 
    solved by computers:

    \begin{itemize}
        \item Playing a decent game of table tennis (Ping-Pong).
        \item Driving in the center of Cairo, Egypt.
        \item Driving in Victorville, California.
        \item Buying a week's worth of groceries at the market.
        \item Buying a week's worth of groceries on the Web.
        \item Playing a decent game of bridge at a competitive level.
        \item Discovering and proving new mathematical theorems.
        \item Writing an intentionally funny story.
        \item Giving competent legal advice in a specialized area of law.
        \item Translating spoken English into spoken Swedish in real time.
        \item Performing a complex surgical operation.
    \end{itemize}

    For the currently infeasible tasks, try to find out what the difficulties are and predict when, if 
    ever, they will be overcome.
    \\
    \begin{itemize}
        \item Yes, actually just recently there was exhibition match between the best ping-pong 
            playing computer and the best ping-pong players.
        \item Maybe, the Google self-driving car has made huge advances recently. I am sure in a few 
            years this should be possible.
        \item Yes, the Google self-driving car has already accomplished this.
        \item No, a robot has not been able to navigate crowded pedestrian areas.
        \item Yes, this has already been accomplished years ago.
        \item Yes, most card and table games have already been mastered by computers.
        \item Yes, there is a system that can prove mathematical theorems, however, it is limited.
        \item No, while a computer can copy funny ideas it cannot come up with original jokes.
        \item Maybe, for example a system like IBM Watson maybe able to provide some advice like 
            and expert system would. However, legal advice is mostly interpretation especially an 
            areas that have no pre-existing legal precedence.
        \item Yes, Google translate does this now.
        \item Yes, this has been done for some time now.
    \end{itemize}
\end{homeworkProblem}
\clearpage

%----------------------------------------------------------------------------------------

%----------------------------------------------------------------------------------------
%	PROBLEM 3
%----------------------------------------------------------------------------------------

\begin{homeworkProblem}
    2.3 For each of the following assertions, say whether it is true or false and support your 
    answer with examples or counterexamples where appropriate.

    \begin{enumerate}[label=\alph*]
        \item An agent that senses only partial information about the state cannot be perfectly rational.
        \item There exist task environments in which no pure reflex agent can behave rationally.
        \item There exits a task environment in which every agent is rational.
        \item The input to an agent program is the same as the input to the agent function.
        \item Every agent function is implementable by some program/machine combination.
        \item Suppose an agent selects its action uniformly at random from the set of possible actions.
              There exists a deterministic task environment in which this agent is rational.
        \item It is possible for a given agent to be perfectly rational in two distinct task environments.
        \item Every agent is rational in an unobservable environment.
        \item A perfectly rational poker-playing agent never loses.
    \end{enumerate}

    \begin{enumerate}[label=\alph*]
        \item False. Rationality is the capability to make an appropriate judgement based on 
            on the information possessed.
        \item True. Consider any board game, a reflex agent would not know the state of the board, thus 
            its responses would be the same regardless of the state.
        \item True. Consider an environment in which all actions have the same reward.
        \item False. The agent function receives input as the entire percept sequence up to the
            current point. The agent program receives input as the current percept alone.
        \item False. See Halting problem which cannot be implemented.
        \item True. Selecting random is rational.
        \item True. 
        \item False. No the agent can do something dumb.
        \item False. Poker is about chance therefore the agent can lose.
    \end{enumerate}

\end{homeworkProblem}
\clearpage

%----------------------------------------------------------------------------------------

%----------------------------------------------------------------------------------------
%	PROBLEM 4
%----------------------------------------------------------------------------------------

\begin{homeworkProblem}
    2.4 For each of the following activities, give a PEAS description of the task environment 
    and characterize it in terms of properties listed in Section 2.3.2.

    \begin{itemize}
        \item Playing soccer.
        \item Exploring the subsurface oceans of Titan.
        \item Shopping for used AI books on the Internet.
        \item Playing a tennis match.
        \item Practicing tennis against a wall.
        \item Performing a high jump.
        \item Knitting a sweater.
        \item Bidding on an item at an auction
    \end{itemize}

    \begin{itemize}
        \item partially observable, stochastic, sequential, dynamic, continuous, multi-agent.
        \item partially observable, stochastic, sequential, dynamic, continuous, single-agent.
        \item partially observable, deterministic, sequential, state, discrete, single-agent.
        \item fully observable, stochastic, episodic, dynamic, continuous, multi-agent.
        \item fully observable, stochastic, episodic, dynamic, continuous single-agent.
        \item fully observable, stochastic, sequential, static, continuous, single-agent.
        \item fully observable, deterministic, sequential, static, continuous, single-agent.
        \item fully observable, strategic, sequential, static, discrete, multi-agent.
    \end{itemize}

    NOTE: I had a really hard time figuring this question and answer out. I did a lot of Googling 
    and used a variety of sources to help me find these answers.

\end{homeworkProblem}
\clearpage

%----------------------------------------------------------------------------------------

\end{document}
