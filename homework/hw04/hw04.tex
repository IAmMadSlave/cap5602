%----------------------------------------------------------------------------------------
%	PACKAGES AND OTHER DOCUMENT CONFIGURATIONS
%----------------------------------------------------------------------------------------

\documentclass{article}

\usepackage{fancyhdr} % Required for custom headers
\usepackage{lastpage} % Required to determine the last page for the footer
\usepackage{extramarks} % Required for headers and footers
\usepackage{graphicx} % Required to insert images
\usepackage{listings}
\usepackage{color}
\usepackage{xcolor}
\usepackage{caption}
\usepackage{enumitem}
\usepackage{amsmath}
\usepackage{tikz}
\usepackage{forest}
\usetikzlibrary{calc,shapes.multipart,chains,arrows}

\DeclareCaptionFont{white}{\color{white}}
\DeclareCaptionFormat{listing}{%
\parbox{\textwidth}{\colorbox{gray}{\parbox{\textwidth}{#1#2#3}}\vskip-4pt}}
\captionsetup[lstlisting]{format=listing,labelfont=white,textfont=white}

% Margins
\topmargin=-0.45in
\evensidemargin=0in
\oddsidemargin=0in
\textwidth=6.5in
\textheight=9.0in
\headsep=0.25in 

\linespread{1.1} % Line spacing

% Set up the header and footer
\pagestyle{fancy}
\lhead{\hmwkAuthorName} % Top left header
\chead{\hmwkClass\ (\hmwkClassInstructor\ \hmwkClassTime): \hmwkTitle} % Top center header
\rhead{\hmwkDueDate} % Top right header
\lfoot{\lastxmark} % Bottom left footer
\cfoot{} % Bottom center footer
\rfoot{Page\ \thepage\ of\ \pageref{LastPage}} % Bottom right footer
\renewcommand\headrulewidth{0.4pt} % Size of the header rule
\renewcommand\footrulewidth{0.4pt} % Size of the footer rule

\setlength\parindent{0pt} % Removes all indentation from paragraphs

%----------------------------------------------------------------------------------------
%	DOCUMENT STRUCTURE COMMANDS
%	Skip this unless you know what you're doing
%----------------------------------------------------------------------------------------

\setcounter{secnumdepth}{0} % Removes default section numbers
\newcounter{homeworkProblemCounter} % Creates a counter to keep track of the number of problems

\newcommand{\homeworkProblemName}{}
\newenvironment{homeworkProblem}[1][Problem \arabic{homeworkProblemCounter}]{ % Makes a new environment called homeworkProblem which takes 1 argument (custom name) but the default is "Problem #"
\stepcounter{homeworkProblemCounter} % Increase counter for number of problems
\renewcommand{\homeworkProblemName}{#1} % Assign \homeworkProblemName the name of the problem
\section{\homeworkProblemName} % Make a section in the document with the custom problem count
}

\forestset{qtree/.style={for tree={
                    parent anchor=south, child anchor=north,
            align=center, inner sep=1pt}},
        .style={qtree}}

%----------------------------------------------------------------------------------------
%   COLORS AND LANGUAGAGE
%----------------------------------------------------------------------------------------

\lstset{
    frame=lrb,xleftmargin=\fboxsep,xrightmargin=-\fboxsep,language=Java,basicstyle=\ttfamily,
    breaklines=true,columns=fullflexible,keepspaces=true,escapeinside={\%*}{*)}
       }

%----------------------------------------------------------------------------------------
%	NAME AND CLASS SECTION
%----------------------------------------------------------------------------------------

\newcommand{\hmwkTitle}{Homework\ \#04} % Assignment title
\newcommand{\hmwkDueDate}{Tuesday,\ September\ 22,\ 2015} % Due date
\newcommand{\hmwkClass}{CAP\ 5602} % Course/class
\newcommand{\hmwkClassTime}{5:00pm} % Class/lecture time
\newcommand{\hmwkClassInstructor}{Finlayson} % Teacher/lecturer
\newcommand{\hmwkAuthorName}{Musa V. Ahmed} % Your name

%----------------------------------------------------------------------------------------

\begin{document}
\belowcaptionskip=-10pt

%----------------------------------------------------------------------------------------
%	PROBLEM 1
%----------------------------------------------------------------------------------------

\begin{homeworkProblem}
    \textbf{6.1} How many solutions are there for the map-coloring problem in
    Figure 6.1? How many solutions if four colors are allowed? Two colors?


\end{homeworkProblem}
\clearpage

%----------------------------------------------------------------------------------------

%----------------------------------------------------------------------------------------
%	PROBLEM 2
%----------------------------------------------------------------------------------------

\begin{homeworkProblem}
    \textbf{6.2} Consider the problem of placing $k$ knights on an $n\times n$
    chessboard such that no two knights are attacking each other, where $k$ is
    given an $k\leq n^2$.

    \begin{enumerate}[label=\alph*.]
        \item Choose a CSP formulation. In your formulation, what are the
            variables?
        \item What are the possible values of each variable?
        \item What sets of variables are constrained, and how?
        \item Now consider the problem of putting \emph{as many knights as
            possible} on the board without any attacks. Explain how to solve
            this with local search by defining appropriate \verb!ACTIONS! and
            \verb!RESULT! functions and a sensible objective function. 
    \end{enumerate}


\end{homeworkProblem}
\clearpage

%----------------------------------------------------------------------------------------

%----------------------------------------------------------------------------------------
%	PROBLEM 3
%----------------------------------------------------------------------------------------

\begin{homeworkProblem}
    \textbf{6.11} Use the AC-3 algorithm to show that arc consistency can
    detect the inconsistency of the partial assignment $\{WA=green,V=red\}$ for
    the problem shows in Figure 6.1.


\end{homeworkProblem}
\clearpage

%----------------------------------------------------------------------------------------

%----------------------------------------------------------------------------------------
%	PROBLEM 4
%----------------------------------------------------------------------------------------

\begin{homeworkProblem}
    \textbf{7.6} Prove, or find a counterexample to, each of the following
    assertions:
    \begin{enumerate}[label=\alph*.]
        \item If $\alpha \models \gamma$ or $\beta \models \gamma$ (or both)
            then $(\alpha \wedge \beta) \models \gamma$
        \item If $\alpha \models (\beta \wedge \gamma)$ then $\alpha \models
            \beta$ and $\alpha \models \gamma$
        \item If $\alpha \models (\beta \vee \gamma)$ then $\alpha \models
            \beta$ or $\alpha \models \gamma$ (or both)
    \end{enumerate}


\end{homeworkProblem}
\clearpage

%----------------------------------------------------------------------------------------

%----------------------------------------------------------------------------------------
%	PROBLEM 5
%----------------------------------------------------------------------------------------

\begin{homeworkProblem}
    \textbf{7.18} Consider the following sentence:\\
    $[(Food \Rightarrow Party)\vee (Drinks \Rightarrow Party)]\Rightarrow
    [(Food \wedge Drinks)\Rightarrow Party]$

    \begin{enumerate}[label=\alph*.]
        \item Determine, using enumeration, whether this sentence is valid,
            satisfiable (but not valid), or unsatisfiable
        \item Convert the left-hand and right-hand sides of the main
            implication into CNF, showing each step, and explaining how the
            results confirm your answer to (a)
        \item Prove your answer to (a) using resolution.
    \end{enumerate}


\end{homeworkProblem}
\clearpage

%----------------------------------------------------------------------------------------


\end{document}
